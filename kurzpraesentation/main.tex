\documentclass{beamer}
\usepackage[utf8]{inputenc}
\usepackage{ngerman}
\usepackage{graphicx}

\usetheme{Warsaw}
\useinnertheme{circles}
\useoutertheme{smoothbars}
\setbeamercovered{transparent}

\title{Journal Club: \emph{Hex-o-Spell}}
\titlegraphic{\includegraphics[width=1.5cm]{img/TUG}}
\author{Stefan Nöhmer}
\date{06.12.2010}


\begin{document}

	\frame{\titlepage}
	
	\begin{frame}{Paper}
	  
	  \begin{block}{\textbf{A Note on Brain Actuated Spelling with the Berlin Brain-Computer Interface}}
	    B. Blankertz, M. Krauledat, G. Dornhege, John Williamson, Roderick Murray-Smith, K.-R. Müller
	    
	    Universal Access in Human-Computer Interaction. Ambient Interaction
	    
      4th International Conference on Universal Access in Human-Computer Interaction, UAHCI 2007
      
      Lecture Notes in Computer Science (Volume 4555, 2007)
    \end{block}
  \end{frame}
	
	\section{Inhaltsübersicht}
	\frame{\tableofcontents}
	
	
	\section[Brain Actuated Spelling]{A Note on Brain Actuated Spelling with the Berlin Brain-Computer Interface}
		
		\subsection{Einführung}
		\begin{frame}[<+->]{BCIs und Buchstabieren}
			\begin{block}{}
				\begin{itemize}
					\item wichtige Anwendung neben Kontrolle von Prothesen
					\item bereits mehrere funktionierende Varianten
					\item Thübingen: binäre Suche (ca. 0.5 Zeichen/min)
					\item Graz: binäre Suche, verändertes BCI (ca. 1 Zeichen/min)
					\item 2 parallele Buchstabenströme (über 2 Zeichen/min)
					\item 6x6-Matrix, blinkende Zeilen/Spalten (bis 6 Zeichen/min)
				\end{itemize}
			\end{block}
		\end{frame}
		
		\begin{frame}[<+->]{Hex-o-Spell}
		  \begin{block}{}
		    \begin{itemize}
		      \item ursprüngliche Idee: Hex-System (PDA)
		      \item buchstabenweiser Aufbau von Wörtern \& Sätzen
		      \item Buchstaben werden nach deren Wahrscheinlichkeit angeordnet
		    \end{itemize}
		  \end{block}
		\end{frame}

		\subsection{Methoden}
		\begin{frame}[<+->]{Berlin Brain-Computer Interface \emph{BBCI}}
		  \begin{block}{}
		    \begin{itemize}
		      \item erkennt Änderungen im EEG
		      \item ausgelöst durch die Vorstellungen von Bewegungen
		      \item hier: Unterscheidung Hand-/Fußbewegung
		    \end{itemize}
		  \end{block}
		\end{frame}
		
		\begin{frame}[<+->]{Buchstabeneingabe}
		  \begin{block}{}
		    \begin{itemize}
		      \item dargestellt durch 6 Sechsecke
		      \item pro Sechseck 5 Buchstaben (inkl. Backspace etc.)
		      \item Auswahl eines Sechsecks durch Pfeil in der Mitte
		      \item mögliche Bewegungen: Pfeil drehen/Pfeil verlängern
		      \item nächster Schritt: pro Sechseck nur mehr 1 Buchstabe
		    \end{itemize}
		  \end{block}
		  
		  \begin{center}
	      \includegraphics[width=0.80\textwidth]{img/hexospell.png}
	      \label{fig:hexospell}
      \end{center}
		\end{frame}
		
		\begin{frame}[<+->]{Buchstabenanordnung}
		  \begin{block}{}
		    \begin{itemize}
		      \item Buchstaben werden nach ihrer Wahrscheinlichkeit angeordnet
		      \item dadurch steigt Eingabegeschwindigkeit
		      \item Berechnung der WS aufgrund der vorangegangenen Buchstaben des Wortes
		    \end{itemize}
		  \end{block}
		\end{frame}
		
		\subsection{Ergebnisse}
		\begin{frame}[<+->]{Demonstration auf der Cebit}
		  \begin{block}{}
		    \begin{itemize}
		      \item Live-Demonstration auf der Cebit 2006
		      \item 2 Probanden gleichzeitig
		      \item bis zu 5 Zeichen/min bzw. über 7 Zeichen/min
		      \item einwandfreie Funktion trotz Störquellen in Umgebung
		    \end{itemize}
		  \end{block}
		\end{frame}
		
				
  \section{Aufbau des Papers}
    
    \subsection{Aufbau des Papers}
    \begin{frame}[<+->]{Aufbau des Papers}
      \begin{block}{}
        \begin{itemize}
          \item \textbf{Abstract:}
            kurze Beschreibung des gesamten Papers
          
          \item \textbf{Introduction:} 
            Erklärung BCIs, bisherige Texteingabemethoden, Idee von Hex-o-Spell
            
          \item \textbf{Methods:}
            Funktionsweise BBCI, Bedienung, Eingabe von Buchstaben, Sprachmodell, Design von Hex-o-Spell
            
          \item \textbf{Results:}
            Funktion bei einer Live-Demonstration, erreichte Geschwindigkeit
            
          \item \textbf{Discussion:}
            Vergleich mit anderen Verfahren, Vorteile von Hex-o-Spell
            
          \item \textbf{References:}
            alle Referenzen aus dem Paper
        \end{itemize}
      \end{block}
    \end{frame}
  
				
		% --- Ende ---
		\begin{frame}
			\begin{center}
				\begin{large}
					Danke für die Aufmerksamkeit!
				\end{large}
			\end{center}
		\end{frame}

\end{document}
